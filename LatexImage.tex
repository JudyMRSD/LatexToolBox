\documentclass{article}
\usepackage[utf8]{inputenc}
\usepackage{amssymb}
\usepackage{amsmath}
\usepackage{float}
\usepackage{subfloat}
\usepackage{graphicx}
\graphicspath{ {images/} }

\newtheorem{theorem}{Theorem}

\title{FFT writeup}
\author{Abhijat Biswas, Sam Clarke, Travers Rhodes, Jin Zhu}
\date{October 2017}

\begin{document}

\maketitle


\section{Example calculation}

1.Flow graph (butterfly diagram) of the complete decomposition of an four-point DFT to show terms that can be reused.
2.Red lines: even and odd terms that make up H(0)
3. Orange lines: Even terms (h(0) and h(2)) that are used to calculate term A can be looked as ``even of the even'' points and ``odd of the even'' points. Similaryly, the odd terms (h(1) and h(3)) are used for calculating term C, and h(1) is the ``even of the odd'' point, and h(2) is the ``odd of the odd'' point.


%Images 
\begin{figure}[H]
  \makebox[\textwidth]{\includegraphics[scale=0.5]{Butterfly.jpg}}
  \caption{Four point butterfly diagram}
  \label{fig:butterfly_all}
\end{figure}

\begin{figure}[H]
  \makebox[\textwidth]{\includegraphics[scale=0.5]{ButterflyH0.jpg}}
  \caption{Terms used for calculating $H_0$}
  \label{fig:butterfly_h0}
\end{figure}

\begin{figure}[H]
  \makebox[\textwidth]{\includegraphics[scale=0.5]{Butterfly_ABCD.jpg}}
  \caption{Terms used for calculating A and C}
  \label{fig:butterfly_abcdTerms}
\end{figure}


%Images in the same line
\begin{figure}[H]
  \makebox[\textwidth]{\includegraphics[width=0.7\textwidth]{Butterfly.jpg}
   \hfill
   \includegraphics[width=0.7\textwidth]{Butterfly.jpg}}
  \caption{plot of sinh(x) over interval [-2,2].}
  \label{fig:question1b}
\end{figure}




\end{document}
